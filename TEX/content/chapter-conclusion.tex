% !TEX root = ../main.tex
%
\chapter{Conclusion}
\label{sec:conclusion}

This research work shows the benefits of a multi-faceted approach to long-term monitoring and analysis of GSI systems.
The integration of accurate data collection, robust data storage, and flexible yet relatable analysis methods across many sites has the potential to enable new insights into design and construction practices and reduce maintenance costs.
The work outlined herein lays the groundwork for carrying out long-term monitoring by addressing common instrumentation issues and potential solutions, describing a flexible framework for data storage, and preliminary work on data-driven performance indicators.

Ensuring that sensor equipment is well suited to conduct appropriate measurements and report valid data allows for an accurate, uninterrupted collection of data.
Understanding the intricacies of sensor configurations, communications, randomness of hydraulic characteristics, and opportunities for error helps mitigate invalid data.
Sensor, wiring schematic, and data logger choices should reflect these nuanced factors.
Digital communications significantly increase a sensor's ability to report valid data, ensuring that measurements taken in one location are correctly captured and stored by a physically separated data logger.
The increased confidence in data collected will enable future analysis of long-term trends, so investments and efforts made now in upgrading systems can have lasting benefits for GSI studies across all sites.
Furthermore, the improved data quality will free up researcher resources that can be focused on deriving insights and analyzing the data.

The FAIR principles are an easy check for data accessibility and make transferring analyses between sites simpler.
A flexible data format enables analyses that are easily scalable, since consistent metadata and controlled vocabulary are used across all sites.
This flexibility enables the expansion of the scope of research that is able to be conducted on larger GSI systems.
This is accomplished through controlled vocabulary for describing data and consistent, yet flexible, formatting of the database schema.
The long format of the main data table allows nearly infinite additional metadata combinations, meaning that future projects and sites will not require any updates to the schema.
This flexibility to add future sites is a key feature of the SIDM, as it enables long term studies to encompass both historical and future data easily in a single repository.
Future work to directly integrate monitoring equipment with the data ingestion process could mean that data becomes available in the SIDM almost instantaneously through the use of IoT protocols and methodologies, greatly expanding the value of consistent formatting and data descriptions.
The SIDM will unlock the potential for expanded collaboration between researchers and among project partners such as PennDOT and AECOM, better understandings of long-term monitoring results, and more robust analyses of multi-site GSI systems.

Lastly, the analyses outlined within are well suited to leveraging the data, as the results are consistent with labor intensive field and lab tests commonly accepted for infiltration capacity.
Recession and infiltration KPIs can be thought of as a proxy to these more complicated tests, such as the single and double ring infiltrometer or Saturo constant head tests, such that the long-term monitoring data is able to act as a substitute for routine GSI health checks.
The results of these health checks, which can be performed remotely by virtue of the connected monitoring network, can be used to determine the need for maintenance at a particular site or the need for more thorough inspection in cases of negatively trending performance.
The data have shown that performance is most highly impacted by temperature, which oscillates with an annual seasonal period.
This relationship is expected due to the change in water viscosity across the range of common temperatures in Philadelphia.
In general, GSI health is related to the recession and drying rates of the ponding level and soil moisture, respectively.
Both relationships are positive, such that higher rates indicate faster transmission of water through the engineered media layers and faster reduction in soil moisture leading to rapid recovery.
Results show that average recession rates between 40 and 120 mm/hr can be expected at the case study site, depending on time of year and water temperature.
Drying rate averages as high as 9.5 cm/hr over the course of a storm can be expected, although this value would be expected to vary between GSI systems based on soil design and pre-storm conditions.

These performance indicators are independent from GSI characteristics such as surface area, loading ratio, or number of inlets or outlets, and are therefore easily transferrable to other sites regardless of configuration.
The KPIs can be calculated for any GSI system, or part of a system, to determine if infiltration rates and recovery rates fall within the expected range, aiding in the design process by enabling feedback regarding target performance.
Few sensors are necessary for performing these analyses, namely a pressure transducer for ponding depth and a pair of soil moisture level sensors at a known separation distance, which helps keep costs for this kind of monitoring low.
The use of a consistent data format for a collection of GSI systems will allow these KPIs to provided up-to-date and at-a-glance information about the health of an entire system, making monitoring and managing maintenance requirements much more simple.
Therefore, implementing a basic monitoring network at many sites is economical, as robust data storage and analysis practices allow for data-driven decisions about allocation of maintenance and repair capital based on near real-time data, reduce wasted time and resources, and enable insight into long-term system-wide trends.