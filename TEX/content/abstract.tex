% !TEX root = ../main.tex
%

\pdfbookmark[0]{Abstract}{Abstract}
\addchap*{Abstract}
\label{sec:abstract}

The need for reliable, transferrable, and broadly accepted performance metrics for green stormwater infrastructure (GSI) is widely discussed across the industry.
The present research is a case study of instrumentation practices, data storage, and a proposal for two site-agnostic performance indicators because they do not account for site parameters.
FAIR (Findable, Accessible, Interoperable, Reusable) principles for data storage and access ensure that historical data and future data are collected and organized in a manner that allows for continued study while maintaining flexibility and controlled vocabulary for robust data descriptions.
Recession rate and infiltration rate are widely understood to be good indicators of GSI performance, but are difficult to test, rely on manual labor by onsite personnel, and test only discrete points within a site.
Continuous monitoring data can be used to approximate a site's average recession and infiltration rates using just a few sensors.
Analyzing trends in the average of these rates on a storm-event basis gives insight into performance trends.
A relationship between recession rate and temperature was found to exist, with a correlation coefficient of 0.735, which has been found in previous studies.
At the case study site, recession rates between 40 and 120 mm/hr can be expected, depending on time of year and water temperature, and average infiltration drying rates as high as 9.5 cm/hr can be expected, although further study and comparison is needed to reduce the high variance seen in these results.
A network of reliable sensors at spatially distributed GSI sites will allow for near-real-time data-driven decisions about maintenance and repair resource allocation, in addition to the analysis of long-term trends in performance.
